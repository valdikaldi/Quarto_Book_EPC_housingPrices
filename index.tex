% Options for packages loaded elsewhere
\PassOptionsToPackage{unicode}{hyperref}
\PassOptionsToPackage{hyphens}{url}
\PassOptionsToPackage{dvipsnames,svgnames,x11names}{xcolor}
%
\documentclass[
  letterpaper,
  DIV=11,
  numbers=noendperiod]{scrreprt}

\usepackage{amsmath,amssymb}
\usepackage{iftex}
\ifPDFTeX
  \usepackage[T1]{fontenc}
  \usepackage[utf8]{inputenc}
  \usepackage{textcomp} % provide euro and other symbols
\else % if luatex or xetex
  \usepackage{unicode-math}
  \defaultfontfeatures{Scale=MatchLowercase}
  \defaultfontfeatures[\rmfamily]{Ligatures=TeX,Scale=1}
\fi
\usepackage{lmodern}
\ifPDFTeX\else  
    % xetex/luatex font selection
\fi
% Use upquote if available, for straight quotes in verbatim environments
\IfFileExists{upquote.sty}{\usepackage{upquote}}{}
\IfFileExists{microtype.sty}{% use microtype if available
  \usepackage[]{microtype}
  \UseMicrotypeSet[protrusion]{basicmath} % disable protrusion for tt fonts
}{}
\makeatletter
\@ifundefined{KOMAClassName}{% if non-KOMA class
  \IfFileExists{parskip.sty}{%
    \usepackage{parskip}
  }{% else
    \setlength{\parindent}{0pt}
    \setlength{\parskip}{6pt plus 2pt minus 1pt}}
}{% if KOMA class
  \KOMAoptions{parskip=half}}
\makeatother
\usepackage{xcolor}
\setlength{\emergencystretch}{3em} % prevent overfull lines
\setcounter{secnumdepth}{5}
% Make \paragraph and \subparagraph free-standing
\makeatletter
\ifx\paragraph\undefined\else
  \let\oldparagraph\paragraph
  \renewcommand{\paragraph}{
    \@ifstar
      \xxxParagraphStar
      \xxxParagraphNoStar
  }
  \newcommand{\xxxParagraphStar}[1]{\oldparagraph*{#1}\mbox{}}
  \newcommand{\xxxParagraphNoStar}[1]{\oldparagraph{#1}\mbox{}}
\fi
\ifx\subparagraph\undefined\else
  \let\oldsubparagraph\subparagraph
  \renewcommand{\subparagraph}{
    \@ifstar
      \xxxSubParagraphStar
      \xxxSubParagraphNoStar
  }
  \newcommand{\xxxSubParagraphStar}[1]{\oldsubparagraph*{#1}\mbox{}}
  \newcommand{\xxxSubParagraphNoStar}[1]{\oldsubparagraph{#1}\mbox{}}
\fi
\makeatother

\usepackage{color}
\usepackage{fancyvrb}
\newcommand{\VerbBar}{|}
\newcommand{\VERB}{\Verb[commandchars=\\\{\}]}
\DefineVerbatimEnvironment{Highlighting}{Verbatim}{commandchars=\\\{\}}
% Add ',fontsize=\small' for more characters per line
\usepackage{framed}
\definecolor{shadecolor}{RGB}{241,243,245}
\newenvironment{Shaded}{\begin{snugshade}}{\end{snugshade}}
\newcommand{\AlertTok}[1]{\textcolor[rgb]{0.68,0.00,0.00}{#1}}
\newcommand{\AnnotationTok}[1]{\textcolor[rgb]{0.37,0.37,0.37}{#1}}
\newcommand{\AttributeTok}[1]{\textcolor[rgb]{0.40,0.45,0.13}{#1}}
\newcommand{\BaseNTok}[1]{\textcolor[rgb]{0.68,0.00,0.00}{#1}}
\newcommand{\BuiltInTok}[1]{\textcolor[rgb]{0.00,0.23,0.31}{#1}}
\newcommand{\CharTok}[1]{\textcolor[rgb]{0.13,0.47,0.30}{#1}}
\newcommand{\CommentTok}[1]{\textcolor[rgb]{0.37,0.37,0.37}{#1}}
\newcommand{\CommentVarTok}[1]{\textcolor[rgb]{0.37,0.37,0.37}{\textit{#1}}}
\newcommand{\ConstantTok}[1]{\textcolor[rgb]{0.56,0.35,0.01}{#1}}
\newcommand{\ControlFlowTok}[1]{\textcolor[rgb]{0.00,0.23,0.31}{\textbf{#1}}}
\newcommand{\DataTypeTok}[1]{\textcolor[rgb]{0.68,0.00,0.00}{#1}}
\newcommand{\DecValTok}[1]{\textcolor[rgb]{0.68,0.00,0.00}{#1}}
\newcommand{\DocumentationTok}[1]{\textcolor[rgb]{0.37,0.37,0.37}{\textit{#1}}}
\newcommand{\ErrorTok}[1]{\textcolor[rgb]{0.68,0.00,0.00}{#1}}
\newcommand{\ExtensionTok}[1]{\textcolor[rgb]{0.00,0.23,0.31}{#1}}
\newcommand{\FloatTok}[1]{\textcolor[rgb]{0.68,0.00,0.00}{#1}}
\newcommand{\FunctionTok}[1]{\textcolor[rgb]{0.28,0.35,0.67}{#1}}
\newcommand{\ImportTok}[1]{\textcolor[rgb]{0.00,0.46,0.62}{#1}}
\newcommand{\InformationTok}[1]{\textcolor[rgb]{0.37,0.37,0.37}{#1}}
\newcommand{\KeywordTok}[1]{\textcolor[rgb]{0.00,0.23,0.31}{\textbf{#1}}}
\newcommand{\NormalTok}[1]{\textcolor[rgb]{0.00,0.23,0.31}{#1}}
\newcommand{\OperatorTok}[1]{\textcolor[rgb]{0.37,0.37,0.37}{#1}}
\newcommand{\OtherTok}[1]{\textcolor[rgb]{0.00,0.23,0.31}{#1}}
\newcommand{\PreprocessorTok}[1]{\textcolor[rgb]{0.68,0.00,0.00}{#1}}
\newcommand{\RegionMarkerTok}[1]{\textcolor[rgb]{0.00,0.23,0.31}{#1}}
\newcommand{\SpecialCharTok}[1]{\textcolor[rgb]{0.37,0.37,0.37}{#1}}
\newcommand{\SpecialStringTok}[1]{\textcolor[rgb]{0.13,0.47,0.30}{#1}}
\newcommand{\StringTok}[1]{\textcolor[rgb]{0.13,0.47,0.30}{#1}}
\newcommand{\VariableTok}[1]{\textcolor[rgb]{0.07,0.07,0.07}{#1}}
\newcommand{\VerbatimStringTok}[1]{\textcolor[rgb]{0.13,0.47,0.30}{#1}}
\newcommand{\WarningTok}[1]{\textcolor[rgb]{0.37,0.37,0.37}{\textit{#1}}}

\providecommand{\tightlist}{%
  \setlength{\itemsep}{0pt}\setlength{\parskip}{0pt}}\usepackage{longtable,booktabs,array}
\usepackage{calc} % for calculating minipage widths
% Correct order of tables after \paragraph or \subparagraph
\usepackage{etoolbox}
\makeatletter
\patchcmd\longtable{\par}{\if@noskipsec\mbox{}\fi\par}{}{}
\makeatother
% Allow footnotes in longtable head/foot
\IfFileExists{footnotehyper.sty}{\usepackage{footnotehyper}}{\usepackage{footnote}}
\makesavenoteenv{longtable}
\usepackage{graphicx}
\makeatletter
\def\maxwidth{\ifdim\Gin@nat@width>\linewidth\linewidth\else\Gin@nat@width\fi}
\def\maxheight{\ifdim\Gin@nat@height>\textheight\textheight\else\Gin@nat@height\fi}
\makeatother
% Scale images if necessary, so that they will not overflow the page
% margins by default, and it is still possible to overwrite the defaults
% using explicit options in \includegraphics[width, height, ...]{}
\setkeys{Gin}{width=\maxwidth,height=\maxheight,keepaspectratio}
% Set default figure placement to htbp
\makeatletter
\def\fps@figure{htbp}
\makeatother
% definitions for citeproc citations
\NewDocumentCommand\citeproctext{}{}
\NewDocumentCommand\citeproc{mm}{%
  \begingroup\def\citeproctext{#2}\cite{#1}\endgroup}
\makeatletter
 % allow citations to break across lines
 \let\@cite@ofmt\@firstofone
 % avoid brackets around text for \cite:
 \def\@biblabel#1{}
 \def\@cite#1#2{{#1\if@tempswa , #2\fi}}
\makeatother
\newlength{\cslhangindent}
\setlength{\cslhangindent}{1.5em}
\newlength{\csllabelwidth}
\setlength{\csllabelwidth}{3em}
\newenvironment{CSLReferences}[2] % #1 hanging-indent, #2 entry-spacing
 {\begin{list}{}{%
  \setlength{\itemindent}{0pt}
  \setlength{\leftmargin}{0pt}
  \setlength{\parsep}{0pt}
  % turn on hanging indent if param 1 is 1
  \ifodd #1
   \setlength{\leftmargin}{\cslhangindent}
   \setlength{\itemindent}{-1\cslhangindent}
  \fi
  % set entry spacing
  \setlength{\itemsep}{#2\baselineskip}}}
 {\end{list}}
\usepackage{calc}
\newcommand{\CSLBlock}[1]{\hfill\break\parbox[t]{\linewidth}{\strut\ignorespaces#1\strut}}
\newcommand{\CSLLeftMargin}[1]{\parbox[t]{\csllabelwidth}{\strut#1\strut}}
\newcommand{\CSLRightInline}[1]{\parbox[t]{\linewidth - \csllabelwidth}{\strut#1\strut}}
\newcommand{\CSLIndent}[1]{\hspace{\cslhangindent}#1}

\KOMAoption{captions}{tableheading}
\makeatletter
\@ifpackageloaded{tcolorbox}{}{\usepackage[skins,breakable]{tcolorbox}}
\@ifpackageloaded{fontawesome5}{}{\usepackage{fontawesome5}}
\definecolor{quarto-callout-color}{HTML}{909090}
\definecolor{quarto-callout-note-color}{HTML}{0758E5}
\definecolor{quarto-callout-important-color}{HTML}{CC1914}
\definecolor{quarto-callout-warning-color}{HTML}{EB9113}
\definecolor{quarto-callout-tip-color}{HTML}{00A047}
\definecolor{quarto-callout-caution-color}{HTML}{FC5300}
\definecolor{quarto-callout-color-frame}{HTML}{acacac}
\definecolor{quarto-callout-note-color-frame}{HTML}{4582ec}
\definecolor{quarto-callout-important-color-frame}{HTML}{d9534f}
\definecolor{quarto-callout-warning-color-frame}{HTML}{f0ad4e}
\definecolor{quarto-callout-tip-color-frame}{HTML}{02b875}
\definecolor{quarto-callout-caution-color-frame}{HTML}{fd7e14}
\makeatother
\makeatletter
\@ifpackageloaded{bookmark}{}{\usepackage{bookmark}}
\makeatother
\makeatletter
\@ifpackageloaded{caption}{}{\usepackage{caption}}
\AtBeginDocument{%
\ifdefined\contentsname
  \renewcommand*\contentsname{Table of contents}
\else
  \newcommand\contentsname{Table of contents}
\fi
\ifdefined\listfigurename
  \renewcommand*\listfigurename{List of Figures}
\else
  \newcommand\listfigurename{List of Figures}
\fi
\ifdefined\listtablename
  \renewcommand*\listtablename{List of Tables}
\else
  \newcommand\listtablename{List of Tables}
\fi
\ifdefined\figurename
  \renewcommand*\figurename{Figure}
\else
  \newcommand\figurename{Figure}
\fi
\ifdefined\tablename
  \renewcommand*\tablename{Table}
\else
  \newcommand\tablename{Table}
\fi
}
\@ifpackageloaded{float}{}{\usepackage{float}}
\floatstyle{ruled}
\@ifundefined{c@chapter}{\newfloat{codelisting}{h}{lop}}{\newfloat{codelisting}{h}{lop}[chapter]}
\floatname{codelisting}{Listing}
\newcommand*\listoflistings{\listof{codelisting}{List of Listings}}
\makeatother
\makeatletter
\makeatother
\makeatletter
\@ifpackageloaded{caption}{}{\usepackage{caption}}
\@ifpackageloaded{subcaption}{}{\usepackage{subcaption}}
\makeatother

\ifLuaTeX
  \usepackage{selnolig}  % disable illegal ligatures
\fi
\usepackage{bookmark}

\IfFileExists{xurl.sty}{\usepackage{xurl}}{} % add URL line breaks if available
\urlstyle{same} % disable monospaced font for URLs
\hypersetup{
  pdftitle={Are home buyers inattentive towards energy efficiency?},
  pdfauthor={Valdimar Einarsson},
  colorlinks=true,
  linkcolor={blue},
  filecolor={Maroon},
  citecolor={Blue},
  urlcolor={Blue},
  pdfcreator={LaTeX via pandoc}}


\title{Are home buyers inattentive towards energy efficiency?}
\usepackage{etoolbox}
\makeatletter
\providecommand{\subtitle}[1]{% add subtitle to \maketitle
  \apptocmd{\@title}{\par {\large #1 \par}}{}{}
}
\makeatother
\subtitle{An econometric analysis based on a new data set created with
NLP and OCR techniques}
\author{Valdimar Einarsson}
\date{2024-07-22}

\begin{document}
\maketitle

\renewcommand*\contentsname{Table of contents}
{
\hypersetup{linkcolor=}
\setcounter{tocdepth}{2}
\tableofcontents
}

\bookmarksetup{startatroot}

\chapter*{Welcome}\label{welcome}
\addcontentsline{toc}{chapter}{Welcome}

\markboth{Welcome}{Welcome}

On this site, I present my empirical project, conducted as part of my
Master's thesis in Applied Economics and Data Science. In this project,
I attempt to determine whether homebuyers in Denmark value energy
efficiency and specific energy-related features in properties.

In particular, I explore whether homebuyers are willing to pay a premium
for energy-efficient properties or, conversely, accept a discount for
properties that are less energy-efficient. To quantify energy efficiency
both the energy label or EPC label (i.e.~the label stated in the Energy
Performance Certificate of each property) and estimated energy
consumption in kWh was utilized.

Furthermore, the impact of different energy heat sources and three
energy systems---specifically, properties with heat pumps, solar panels,
and solar heating---on property prices was examined.

\bookmarksetup{startatroot}

\chapter*{Key languages/tools used}\label{key-languagestools-used}
\addcontentsline{toc}{chapter}{Key languages/tools used}

\markboth{Key languages/tools used}{Key languages/tools used}

\begin{tcolorbox}[enhanced jigsaw, bottomtitle=1mm, colbacktitle=quarto-callout-note-color!10!white, coltitle=black, opacitybacktitle=0.6, colframe=quarto-callout-note-color-frame, colback=white, rightrule=.15mm, arc=.35mm, leftrule=.75mm, breakable, title={Programming Languages and their functions}, left=2mm, titlerule=0mm, toptitle=1mm, opacityback=0, bottomrule=.15mm, toprule=.15mm]

\begin{longtable}[]{@{}llll@{}}
\toprule\noalign{}
\endhead
\bottomrule\noalign{}
\endlastfoot
\multicolumn{2}{@{}l}{%
\includegraphics{assets/python_icon.png} Python} &
\multicolumn{2}{l@{}}{%
\includegraphics{assets/r_icon.png} R} \\
& \href{data.html}{\includegraphics{assets/data-processing_icon_2.png} -
Data cleaning and pre-processing} & &
\includegraphics{assets/data_visualization_icon.png} - Data
visualization \\
& \includegraphics{assets/web_scraping_icon_2.png} - Web scraping & &
\includegraphics{assets/linear-regression_icon.png} - Regression
modelling \\
& \includegraphics{assets/pdf_text_icon.png} - PDF parsing (The EPC
reports) & & \\
& \includegraphics{assets/nlp_icon.png} - Text classification models
(NLP) & & \\
& \includegraphics{assets/ocr_icon.png} - Optical character recognition
(OCR) & & \\
& \includegraphics{assets/geographical-data_icon.png} - Geographical
data extraction & & \\
\end{longtable}

\end{tcolorbox}

\begin{tcolorbox}[enhanced jigsaw, bottomtitle=1mm, colbacktitle=quarto-callout-tip-color!10!white, coltitle=black, opacitybacktitle=0.6, colframe=quarto-callout-tip-color-frame, colback=white, rightrule=.15mm, arc=.35mm, leftrule=.75mm, breakable, title={Data and tool related to each function}, left=2mm, titlerule=0mm, toptitle=1mm, opacityback=0, bottomrule=.15mm, toprule=.15mm]

\end{tcolorbox}

\bookmarksetup{startatroot}

\chapter{Introduction}\label{introduction}

In the last decades, addressing the issue of climate change has become
increasingly urgent. Governments worldwide have embarked on ambitious
plans to lower carbon dioxide emissions and reduce energy consumption
due to this heightened awareness. The European Union has established a
long-term strategy to become a climate-neutral economy with net-zero
greenhouse gas emissions. With this plan, the EU aims to establish a
crucial stepping stone for the world to achieve the goal of the Paris
Agreement by keeping global warming below 2 degrees Celsius. This goal
emphasized the urgent need for substantial economic change to reduce
carbon dioxide emissions, particularly in high-emissions sectors. Among
those sectors is the building industry, which is estimated to be
responsible for 40\% of global carbon dioxide emissions (Carlin 2022).
About 70\% of these emissions are generated by building operations,
while construction activities contribute the remaining 30\%~(Carlin
2022). Regarding the European Union, the European Environment Agency
(2023) published a report stating that the building sector is
responsible for about 35\% of energy-related emissions in the European
Union for the year 2021.These emissions are primarily connected to
activities like the combustion of fossil fuels within buildings, such as
oil and gas boilers for heating, alongside electricity consumption for
lighting, water heating, and cooling systems. In 2002, the EU introduced
the Energy Performance of Buildings Directive (EPBD) with the main
objective of promoting energy efficiency (EC 2002). The European Union
acknowledged the importance of a new policy instrument to tackle the
emissions from the building

EPBD introduced the energy performance certificates (EPC) to tackle both
the information failures, i.e.~the information gap on the energy
efficiency of properties that exists between property owners and
prospective buyers, as well as the behavioral failure, such as
decision-making heuristics and biases. This initiative aims to provide
property buyers with accurate, direct, and cost-free information to
ensure they can make well-informed decisions that would otherwise be
contained by lack of transparency. Additionally, the information that
the EPC reports provide can serve as a motivator for both owners and
builders to invest in energy-efficient measures since it can be
hypothesized that higher property prices and rents can result from a
building's improved energy performance. The Directive did not, however,
compel property owners to publicly reveal the energy label, i.e., their
property's energy performance, in an advert. However, a recast of the EU
Directive on the energy performance of buildings was set in place in
2010. This recast imposed on Member States to make it a requirement that
property owners include the energy performance indicator, i.e.~the
energy performance rating ranging from A to G, in any commercial media
real estate advertisement (EU 2010). Interestingly, between 2005 and
2021, total carbon dioxide emissions from the EU building sector dropped
by 31\% (European Enviroment Agency 2023)

Since 1997, it has been mandatory in Denmark to have an energy
certification. This makes Denmark one of the first countries in the EU
to implement an EPC scheme, which includes benchmarking, an energy
label, and an energy performance rating in the building sector.
Interestingly, even though Denmark became one of the first to implement
the system, they did not observe an impact on the real estate market
until 2011, i.e., one year after the recast of the EPBD (Jensen, Hansen,
and Kragh 2016). The 2010 recast sparked many research studies to
investigate the relationship between the energy performance of
properties and property prices. The majority of the studies that
investigate the EU housing market focus on the energy label. Other
studies focus either on estimated or measured energy consumption. The
results of these studies, however, have been mixed. Some studies
identify significant price premiums associated with high energy
performance labels or low energy consumption, whereas others observe no
premium at all. Overall, the majority of studies identify significant
price premiums, indicating a general trend toward the valuation of
energy-efficient properties (Cespedes-Lopez et al. 2019). Concerning the
Danish housing market, there is only a small amount of existing
literature on the subject where the latest research that was found was
conducted in the year 2016 (see Christensen et al. 2014; Jensen, Hansen,
and Kragh 2016; and Næss-Schmidt, Heebøll, and Fredslund 2015).
Furthermore, only a limited number of studies have taken into account
additional information stated in the EPC reports, such as the renovation
recommendations, renovations cost, and investment cost stated in the
reports or the potential energy label a property can obtain from
implementing recommended renovations.

This study investigates the relationship between energy performance and
property prices by accounting for additional information taken from the
reports. A hedonic model is applied to a unique data set of properties
in Denmark from the years 2010 to 2023. Information is extracted using
regular expressions, OCR, and NLP text classification techniques from
over 776,106 EPC reports and then combined with scraped property
characteristics and transaction data consisting of over 2.2 million
observations, obtained from the website
\href{https://www.boliga.dk/}{boliga.dk}. Once all data is merged and
cleaned, we obtain a final sample consisting of over 728,794
observations.

With this unique dataset, the contributions of this thesis are
threefold. Firstly, it is the first paper to account for all residential
property types in Denmark. To the best of our knowledge, all studies
that investigate the Danish housing market have only focused on detached
houses. Secondly, besides the energy label and energy consumption, we
incorporate additional information from the EPC reports related to
renovation potentials toward energy efficiency. The Idea here is that if
consumers pay attention to the information in the EPC reports and are
willing to pay a price premium for energy-efficient property, hence a
discount for non-energy efficient property, then that premium/discount
might vary depending on the renovation potential and the energy
efficiency that the property can achieve from it. This might play a
significant role in non-energy efficient properties. Finally, we measure
the effect of tighter policy implementation for the Danish housing
market on property renovations. The Danish authorities implemented
tighter building regulations on January 1, 2018, which set minimum
energy performance requirements for all new buildings and minimum energy
efficiency requirements for renovations of existing buildings.

The thesis is structured as follows:
\href{epbd_and_danish_EPC_scheme.qmd}{Chapter 2: EPBD and the Danish EPC
scheme} introduces the implementation of the Energy Performance
Certificates in Denmark and the structure of the reports. Subsequently,
\href{literature_review.qmd}{Chapter 3: Literature Review} looks into
the literature related to this paper.
\href{data/data_housing.qmd}{Chapter 4: Data} chapter gives a detailed
description of the data and how it was obtained.
\href{empirical_strategy/empirical_strategy_EstimationStrategy.qmd}{Chapter
5: Empirical Strategy} outlines the theory of the hedonic model and the
empirical strategy utilized in this paper. Finally, in the
\href{results.qmd}{Chapter 6: Results}, we present the results.

\bookmarksetup{startatroot}

\chapter{EPBD and the Danish EPC
scheme}\label{epbd-and-the-danish-epc-scheme}

With the establishment of the Energy Performance of Buildings Directive
(EPBD) in 2001, the EU constructed its main legislative tool to improve
the energy performance of buildings. With this Directive, EU members
were given a framework to develop regulations for the energy efficiency
of buildings. The framework provided a rounded structure on matters such
as defining energy performance and methodology to quantify and measure
the energy efficiency of buildings (EC 2002). Thus, the EU Commission
highlighted the importance of energy conservation in buildings and
formulated the EPC system. From 2006 onwards, the EPC system was
gradually implemented across member states, with all states mandated to
have an operational system in place by the final deadline of 4 January
2009 (Arcipowska et al. 2014). Since 2002, the EU Commission has updated
the Directive through the years. In 2010, the EU recast the EPBD
(European Directive 2010/31/EU) by updating the previous Directive as
well as improving the quality of the Energy Performance Certification
scheme by implementing additional requirements regarding independent
qualified and/or accredited experts authorised to perform assessments of
a building's energy performance (EU 2010). Moreover, The EU Commission
later expanded the scope of the recast in January 2012 by introducing a
comparative methodology framework to determine the most cost-effective
levels of minimum energy performance standards for both buildings and
building elements. The recast of the EPBD played a vital role in
enhancing transparency in the methodology of the energy performance of
buildings and minimising potential information asymmetries between the
property buyer and seller. Before the recast, sellers were not obligated
to publicly display the energy performance of the property. Sellers
could simply showcase the certificate at the time of signing the
purchase agreement or rental contract. However, this changed with the
recast, which required that when buildings

``are offered for sale or for rent, the energy performance indicator of
the energy performance certificate of the building or the building unit,
as applicable, is stated in the advertisements in commercial media'' (EU
2010, 24)

The fundamental motivations of the recast was to push for more
sustainable development and make critical information more accessible to
potential buyers and renters such that it fosters greater demand for
energy-efficient buildings. By doing so, the goal is to reduce energy
consumption and greenhouse gas emissions in the housing sector. As a
result, EPCs simplify the procedure of comparing similar buildings and
can help boost awareness in the market towards energy efficiency, which
intends to assist the EU in achieving its sustainability goals.
(Arcipowska et al. 2014)

Members of the European Union have a degree of freedom to develop their
measurement systems, accreditation requirements, and EPC layouts under
the subsidiarity principle. Nevertheless, they are required to be in
line with the EU framework policy given in the EPBD and meet the minimum
methodology requirement specified in Annex 1 in the Directive. Certain
key information must be displayed in the EPCs to fulfil EPBD
requirements. This includes the energy performance index, which reflects
the energy efficiency rating of the building, as well as reference
values, such as minimum energy performance requirements, enabling
comparison between different buildings. Expert recommendations are also
required, where possible, and feasible home renovations that can improve
the energy performance of the building are listed. Finally, the
methodology estimate needs to be standardised at the national level. (EU
2010)

A report made by Arcipowska et al. (2014) gives a clear overview of the
implementation of EPC across the EU and the different countries that
implement an EPC scheme. From the report, it can be seen that over the
course of the implementation, the majority of countries have developed a
letter-based rating scheme, for instance, on a scale from A to G, where
A is most efficient and G is least efficient. These ratings are
determined either based on specific energy consumption values or by
comparing the building's performance to reference buildings.
Furthermore, the majority of countries have made it a requirement that
qualified experts must make an on-site visit to the property to issue an
energy performance certificate for existing buildings, that includes
Denmark. The on-site requirement is one of the ways to ensure that the
quality of the input data is upheld for the calculation process.

Since the year 1997, it has been mandatory in Denmark to have an energy
certification. Denmark was the first EU member to construct a complete
building certification scheme, which consisted of an energy performance
label system ranging from labels A to G. The Danish Energy Agency
projected a positive impact on market activity by promoting the sale of
highly energy-efficient buildings from an early implementation.
Additionally, Danish policymakers argued that the implementation of the
energy performance rating for buildings would not only raise the energy
efficiency standards of the building stock, but would also serve as an
important motivator in driving the society's overall efforts towards
reducing carbon emissions (Jensen, Hansen, and Kragh 2016). As a result,
starting from 1 January 1997, every building that was listed for sale
was required to have a valid energy performance rating (Jensen, Hansen,
and Kragh 2016). Interestingly, Jensen, Hansen, and Kragh (2016) point
out in their paper that even though Denmark became one of the first EU
members to implement a scheme, there was no observable effect for 15
years. The Danish authorities implemented the 2010 EBPD recast almost
two weeks after it got passed, i.e.~on 1 July 2010. One year later, an
effect was observed on the market. Danish real estate agents claimed
that the easiest properties to sell were properties with higher energy
performance. (Jensen, Hansen, and Kragh 2016)

The Danish EPC scheme is maintained by the Danish Energy Agency (DEA).
They are responsible for monitoring, maintaining quality assurance, and
conducting further developments on the program. In order to receive an
EPC report for a given building, a property must be inspected by a
licensed energy consultant, which then will determine the building's
energy performance (Concerted Action EPBD 2020). The licensed energy
consultant calculates the building's energy consumption and signs a
label on a scale from A to G, where A indicates the highest energy
standard and G the lowest. He inspects the building's quality concerning
insulation, windows and doors, heating installation system, heating
sources, etc. On this basis, the building's energy consumption is
calculated under certain assumptions such as weather, family size,
operating hours, consumption habits, etc ({``Energimærkning''} n.d.). In
other words, the energy consumption is theoretical. However, it is set
up to reflect normal household consumption, given the building's
attributes. With that being said, it may not necessarily reflect the
actual energy consumption, which is, of course, highly dependent on both
the weather and the habits of the residence. Moreover, the EPC reports
give an overview of the energy improvements that the energy consultants
deem cost-effective to implement at the time of reporting.

Over the years, the energy labels in Denmark have changed multiple
times. Table \textbf{?@tbl-EPC\_table} depicts the conversion from all
the different energy labels that the Danish EPC scheme has had to the
current labelling system, as well as the current official energy
thresholds for each label. To this day, the A label in Denmark is split
into three subcategories: A2020, A2015 and A2010. Where A2020 represents
the most energy-efficient category, and G is labelled the least
efficient. The 2013 scale is the official EPC labels in Denmark today.
To this day, newly constructed buildings must have an energy label A2015
or A2020~({``Hvornår Skal Bygninger Energimærkes?''} 2016). As Table
\textbf{?@tbl-EPC\_table} shows, a house of the size 100 square meters
labelled D has an energy consumption between 167 (135+3200/100) and 217
(175+4200/100) kWh per square meter per year, i.e.~between 16.700 to
21.700 kWh per year. On the other hand, a property of the same size with
the label A2020 has an energy consumption equal to or less than 25 kWh
per square meter per year i.e., 2,500 kWh per year or less.

\begin{longtable}[]{@{}
  >{\centering\arraybackslash}p{(\columnwidth - 10\tabcolsep) * \real{0.1222}}
  >{\centering\arraybackslash}p{(\columnwidth - 10\tabcolsep) * \real{0.1222}}
  >{\centering\arraybackslash}p{(\columnwidth - 10\tabcolsep) * \real{0.1222}}
  >{\centering\arraybackslash}p{(\columnwidth - 10\tabcolsep) * \real{0.1222}}
  >{\centering\arraybackslash}p{(\columnwidth - 10\tabcolsep) * \real{0.2000}}
  >{\centering\arraybackslash}p{(\columnwidth - 10\tabcolsep) * \real{0.2667}}@{}}
\caption{Conversion table - EPC classification for residential
buildings}\tabularnewline
\toprule\noalign{}
\begin{minipage}[b]{\linewidth}\centering
2006\\
scale\strut
\end{minipage} & \begin{minipage}[b]{\linewidth}\centering
2008V1\\
scale\strut
\end{minipage} & \begin{minipage}[b]{\linewidth}\centering
2008V2\\
scale\strut
\end{minipage} & \begin{minipage}[b]{\linewidth}\centering
2011\\
scale\strut
\end{minipage} & \begin{minipage}[b]{\linewidth}\centering
2013\\
scale\\
(Current scale)\strut
\end{minipage} & \begin{minipage}[b]{\linewidth}\centering
Threshold limit\\
2013-\\
\((kWh/m^2/year)\)\strut
\end{minipage} \\
\midrule\noalign{}
\endfirsthead
\toprule\noalign{}
\begin{minipage}[b]{\linewidth}\centering
2006\\
scale\strut
\end{minipage} & \begin{minipage}[b]{\linewidth}\centering
2008V1\\
scale\strut
\end{minipage} & \begin{minipage}[b]{\linewidth}\centering
2008V2\\
scale\strut
\end{minipage} & \begin{minipage}[b]{\linewidth}\centering
2011\\
scale\strut
\end{minipage} & \begin{minipage}[b]{\linewidth}\centering
2013\\
scale\\
(Current scale)\strut
\end{minipage} & \begin{minipage}[b]{\linewidth}\centering
Threshold limit\\
2013-\\
\((kWh/m^2/year)\)\strut
\end{minipage} \\
\midrule\noalign{}
\endhead
\bottomrule\noalign{}
\endlastfoot
- & - & - & - & A2020 & \(\le 25\) \\
A1 & - & - & A1 & A2015 & \(\le 41.0 + 1000/A\) \\
A2 & A & A1,A2 & A2 & A2010 & \(\le 71.3 + 1650/A\) \\
B1 & B & B & B & B & \(\le 95.0 + 2200/A\) \\
B2,C1 & C & C & C & C & \(\le 135+ 3200/A\) \\
C2,D1 & D & D & D & D & \(\le 175 + 4200/A\) \\
D2,E1 & E & E & E & E & \(\le 215 + 5200/A\) \\
E2,F1 & F & F & F & F & \(\le 265 + 6500/A\) \\
F2,G1,G2 & G & G & G & G & \(> 265 + 6500/A\) \\
\end{longtable}

As mentioned earlier, this thesis will analyse the period 1. July 2010
to 2023. To quantify the label effects across the whole sampling period,
all labels from older labeling systems are converted into the newest
system based on the conversion table. Furthermore, within the period of
this study, the layout and design of the Danish EPC reports have had
three major changes. This is important because not all designs convey
the exact same information. For clarity, the designs of the certificates
will be referred to as Design 1, Design 2, and Design 3, where Design 1
is the oldest and Design 3 is the newest. The biggest difference is
between the first design and the other designs. Essentially the
renovation recommendation includes the following information: the type
of renovation, estimated investment cost, and energy cost saved. In all
designs, the renovations are categorised into two groups: profitable
renovations and other renovations. The first group represents so-called
``profitable renovations,'' i.e., renovations that the licensed energy
expert deems profitable based on estimated costs and savings. A
renovation is deemed profitable if the energy savings can pay back the
investment before the proposed renovation needs to be replaced again. As
an example, if the proposal is to replace a circulation pump for hot
water, the pump is expected to last for 15 years, and the savings
proposal is considered profitable if the estimated energy savings can
pay back the investment over 15 years. The second group will be referred
to as the ``non-profitable renovations'', which represent renovation
proposals that cannot repay the investment before the proposed
renovation needs to be replaced again. However, these renovations are
often beneficial to consider if the building is being renovated or if
there are building components that need to be replaced anyway. The
profitable renovations include both investment cost, energy savings in
Danish krone (DKK), and energy consumption saved. For non-profitable
renovations, only the energy savings in DKK and the energy consumption
saved are included. In addition to this, both EPC designs 2 and 3
include the energy label that a building would get from implementing
profitable renovations and the energy label it would get from
implementing all renovations, i.e.~both profitable and non-profitable
renovations. However, design one only includes the potential energy
label the property would get from all renovations. Overall, in
\href{data/data_housing.qmd}{chapter 4: Data} we will go in more detail
how we harmonise our data over all the designs.

\bookmarksetup{startatroot}

\chapter{Literature review}\label{literature-review}

\begin{Shaded}
\begin{Highlighting}[]
\DecValTok{1} \SpecialCharTok{+} \DecValTok{1}
\end{Highlighting}
\end{Shaded}

\begin{verbatim}
[1] 2
\end{verbatim}

\part{Data}

\chapter{Housing data}\label{housing-data}

\begin{Shaded}
\begin{Highlighting}[]
\DecValTok{1} \SpecialCharTok{+} \DecValTok{1}
\end{Highlighting}
\end{Shaded}

\begin{verbatim}
[1] 2
\end{verbatim}

\chapter{Energy performance certificates
data}\label{energy-performance-certificates-data}

\begin{Shaded}
\begin{Highlighting}[]
\DecValTok{1} \SpecialCharTok{+} \DecValTok{1}
\end{Highlighting}
\end{Shaded}

\begin{verbatim}
[1] 2
\end{verbatim}

\chapter{Geographical data}\label{geographical-data}

\begin{Shaded}
\begin{Highlighting}[]
\DecValTok{1} \SpecialCharTok{+} \DecValTok{1}
\end{Highlighting}
\end{Shaded}

\begin{verbatim}
[1] 2
\end{verbatim}

\part{Empirical Strategy}

\chapter{Empirical Strategy}\label{empirical-strategy-1}

\begin{Shaded}
\begin{Highlighting}[]
\DecValTok{1} \SpecialCharTok{+} \DecValTok{1}
\end{Highlighting}
\end{Shaded}

\begin{verbatim}
[1] 2
\end{verbatim}

\chapter{Empirical Strategy}\label{empirical-strategy-2}

\begin{Shaded}
\begin{Highlighting}[]
\DecValTok{1} \SpecialCharTok{+} \DecValTok{1}
\end{Highlighting}
\end{Shaded}

\begin{verbatim}
[1] 2
\end{verbatim}

\bookmarksetup{startatroot}

\chapter{Results}\label{results}

\begin{Shaded}
\begin{Highlighting}[]
\DecValTok{1} \SpecialCharTok{+} \DecValTok{1}
\end{Highlighting}
\end{Shaded}

\begin{verbatim}
[1] 2
\end{verbatim}

\bookmarksetup{startatroot}

\chapter{Discussion}\label{discussion}

\begin{Shaded}
\begin{Highlighting}[]
\DecValTok{1} \SpecialCharTok{+} \DecValTok{1}
\end{Highlighting}
\end{Shaded}

\begin{verbatim}
[1] 2
\end{verbatim}

\bookmarksetup{startatroot}

\chapter*{References}\label{references}
\addcontentsline{toc}{chapter}{References}

\markboth{References}{References}

\phantomsection\label{refs}
\begin{CSLReferences}{1}{0}
\bibitem[\citeproctext]{ref-arcipowska_energy_2014}
Arcipowska, Aleksandra, Filippos Anagnostopoulos, Francesco Mariottini,
and Sara Kunkel. 2014. {``Energy Performance Certificates Across the
{EU}.''} Buildings Performance Institute Europe (BPIE).
\url{https://bpie.eu/wp-content/uploads/2015/10/Energy-Performance-Certificates-EPC-across-the-EU.-A-mapping-of-national-approaches-2014.pdf}.

\bibitem[\citeproctext]{ref-carlin_40_2022}
Carlin, David. 2022. {``40\% {Of} {Emissions} {Come} {From} {Real}
{Estate}; {Here}'s {How} {The} {Sector} {Can} {Decarbonize}.''}
\emph{Forbes}, April.
\url{https://www.forbes.com/sites/davidcarlin/2022/04/05/40-of-emissions-come-from-real-estate-heres-how-the-sector-can-decarbonize/}.

\bibitem[\citeproctext]{ref-cespedes-lopez_meta-analysis_2019}
Cespedes-Lopez, Maria-Francisca, Raul-Tomas Mora-Garcia, V. Raul
Perez-Sanchez, and Juan-Carlos Perez-Sanchez. 2019. {``Meta-{Analysis}
of {Price} {Premiums} in {Housing} with {Energy} {Performance}
{Certificates} ({EPC}).''} \emph{Sustainability} 11 (22): 6303.
\url{https://doi.org/10.3390/su11226303}.

\bibitem[\citeproctext]{ref-christensen_energy_2014}
Christensen, Toke Haunstrup, Kirsten Gram-Hanssen, Marjolein de
Best-Waldhober, and Afi Adjei. 2014. {``Energy Retrofits of {Danish}
Homes: Is the {Energy} {Performance} {Certificate} Useful?''}
\emph{Building Research \& Information} 42 (4): 489--500.
\url{https://doi.org/10.1080/09613218.2014.908265}.

\bibitem[\citeproctext]{ref-concerted_action_epbd_implementation_2020}
Concerted Action EPBD. 2020. {``Implementation of the {EPBD} in
{Denmark} (2020).''}
\url{https://www.ca-epbd.eu/Media/638373601209185553/Implementation-of-the-EPBD-in-Denmark--2020.pdf}.

\bibitem[\citeproctext]{ref-ec_eu_2002}
EC. 2002. {``{EU} {Directive} 91/{EC} of the {European} {Parliament} and
of the {Council} of 16 {December} 2002 on the {Energy} {Performance} of
{Buildings}.''} \url{https://eur-lex.europa.eu/eli/dir/2002/91/oj}.

\bibitem[\citeproctext]{ref-noauthor_energimaerkning_nodate}
{``Energimærkning.''} n.d. \emph{Energifocus}. Accessed April 29, 2024.
\url{https://energifocus.dk/p/ydelser/energimaerkning}.

\bibitem[\citeproctext]{ref-eu_eu_2010}
EU. 2010. {``{EU} {Directive}, 31/{EU} of the {European} {Parliament}
and of the {Council} of 19 {May} 2010 on the {Energy} {Performance} of
{Buildings}.''} \url{https://eur-lex.europa.eu/eli/dir/2010/31/oj}.

\bibitem[\citeproctext]{ref-european_enviroment_agency_greenhouse_2023}
European Enviroment Agency. 2023. {``Greenhouse Gas Emissions from
Energy Use in Buildings in {Europe}.''}
\url{https://www.eea.europa.eu/en/analysis/indicators/greenhouse-gas-emissions-from-energy}.

\bibitem[\citeproctext]{ref-noauthor_hvornar_2016}
{``Hvornår Skal Bygninger Energimærkes?''} 2016. \emph{Energistyrelsen}.
\url{https://ens.dk/ansvarsomraader/energimaerkning-af-bygninger/hvornaar-skal-bygninger-energimaerkes}.

\bibitem[\citeproctext]{ref-jensen_market_2016}
Jensen, Ole Michael, Anders Rhiger Hansen, and Jesper Kragh. 2016.
{``Market Response to the Public Display of Energy Performance Rating at
Property Sales.''} \emph{Energy Policy} 93 (June): 229--35.
\url{https://doi.org/10.1016/j.enpol.2016.02.029}.

\bibitem[\citeproctext]{ref-naess-schmidt_homes_2015}
Næss-Schmidt, Sigurd, Christian Heebøll, and Niels C. Fredslund. 2015.
{``Do Homes with Better Energy Efficiency Ratings Have Higher House
Prices? {Econometric} Approach.''} Copenhagen Economics.
\url{https://ens.dk/sites/ens.dk/files/Energibesparelser/bilag_-_do_homes_with_better_energy_efficiency_ratings_have_higher_house_prices_oekonometrisk_tilgang.pdf}.

\end{CSLReferences}

\bookmarksetup{startatroot}

\chapter{Appendix}\label{appendix}

\begin{Shaded}
\begin{Highlighting}[]
\DecValTok{1} \SpecialCharTok{+} \DecValTok{1}
\end{Highlighting}
\end{Shaded}

\begin{verbatim}
[1] 2
\end{verbatim}




\end{document}
